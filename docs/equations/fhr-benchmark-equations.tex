\documentclass[letterpaper,11pt]{article}
\usepackage[top=1.0in,bottom=1.0in,left=1.0in,right=1.0in]{geometry}
\usepackage{verbatim}
\usepackage{amssymb}
\usepackage{graphicx}
\usepackage{longtable}
\usepackage{amsfonts}
\usepackage{amsmath}
\usepackage{hyperref}
\usepackage{float}
\usepackage{setspace}
\usepackage[acronym,toc]{glossaries}  % acronyms inclusion
\usepackage{color,soul}
\makeglossary
\include{acros}
\usepackage{gensymb}
\usepackage{amssymb}
\usepackage{enumitem}
\usepackage{csquotes}

\usepackage{tikz}
\usetikzlibrary{positioning, arrows, decorations, shapes}
\usetikzlibrary{shapes.geometric,arrows}

\definecolor{illiniblue}{HTML}{B1C6E2}
\definecolor{illiniorange}{HTML}{f8c2a2}
\definecolor{pink}{HTML}{e2b1c2}
\definecolor{green}{HTML}{c2e2b1}
\definecolor{purple}{HTML}{b9b1e2}
\tikzstyle{snoblock} = [rectangle, 
text width=5em, text centered,  minimum height=0em]
\tikzstyle{noblock} = [rectangle, 
text width=5em, text centered,  minimum height=3em]
\tikzstyle{loblock} = [rectangle, draw, fill=illiniorange, 
text width=15em, text centered, rounded corners, minimum height=3em]
\tikzstyle{lbblock} = [rectangle, draw, fill=illiniblue, 
text width=15em, text centered, rounded corners, minimum height=3em]
\tikzstyle{oblock} = [rectangle, draw, fill=illiniorange, 
text width=10em, text centered, rounded corners, minimum height=3em]
\tikzstyle{bblock} = [rectangle, draw, fill=illiniblue, 
text width=10em, text centered, rounded corners, minimum height=3em]
\tikzstyle{arrow} = [thick,->,>=stealth]
\tikzstyle{pblock} = [rectangle, draw, fill=pink, 
text width=10em, text centered, rounded corners, minimum height=3em]
\tikzstyle{gblock} = [rectangle, draw, fill=green, 
text width=10em, text centered, rounded corners, minimum height=3em]
\tikzstyle{ppblock} = [rectangle, draw, fill=purple, 
text width=10em, text centered, rounded corners, minimum height=3em]
\tikzstyle{lppblock} = [rectangle, draw, fill=purple, 
text width=15em, text centered, rounded corners, minimum height=3em]
\tikzstyle{arrow} = [thick,->,>=stealth]
\tikzstyle{bbblock} = [rectangle, draw, fill=illiniblue, 
text width=1em, text centered, rounded corners, minimum height=1em]
\tikzstyle{boblock} = [rectangle, draw, fill=illiniorange, 
text width=1em, text centered, rounded corners, minimum height=1em]
\tikzstyle{bpblock} = [rectangle, draw, fill=pink, 
text width=1em, text centered, rounded corners, minimum height=1em]
\tikzstyle{bgblock} = [rectangle, draw, fill=green, 
text width=1em, text centered, rounded corners, minimum height=1em]
\tikzstyle{bppblock} = [rectangle, draw, fill=purple, 
text width=1em, text centered, rounded corners, minimum height=1em]

\author{Gwendolyn J. Chee}

\title{FHR Benchmark Equations }
\begin{document}
\maketitle
\hrulefill
\onehalfspacing

This document contains the assumptions and equations used for the FHR 
benchmark excel spreadsheet results. 

\section{Phase 1a Required Results}
\begin{enumerate}[label=(\alph*)]
    \item Effective multiplication factor 
    \item Reactivity coefficients ($\beta_{eff}$, fuel Doppler coefficient, FLiBe 
    temperature coefficient, graphite temperature coefficient)
    \item Tabulated fission source distribution, at several levels of granularity 
    (by fuel plate, by fuel stripe, by 1/5-th fuel stripe). Optional: visualized fission 
    density distribution.
    \item Visualized distribution of the neutron flux distribution, in 3 coarse energy groups
    \item Neutron spectrum, fuel assembly average. Optional: by region.
\end{enumerate}

\subsection{Effective multiplication factor (a)}
Assumptions made: 
\begin{itemize}
    \item No. of CPUs = No. of Nodes $\times$ 32 (no. of CPUs in each Blue Waters XE node)
    \item CPU-time = No. of CPUs $\times$ Total time in simulation (in openmc's results file)
    \item Wall Clock Time = Total time elapsed (in openmc's results file)
\end{itemize}

\subsection{Reactivity coefficients (b)}
We assume 1 energy group and 6 delayed neutron groups for $\beta_{eff}$. 
\begin{align*}
    \beta_{eff} = \sum_k \beta_k
\end{align*}

Doppler reactivity coefficient (fuel): 
\begin{align*}
    \frac{\Delta \rho}{\Delta T_f} &= 
    \frac{\rho_{1150K}-\rho_{1100K}}{1150-1100} [\frac{pcm}{K}] \\
    \delta \frac{\Delta \rho}{\Delta T_f} &= 
    \frac{\sqrt{\delta (\rho_{1150K})^2+(\delta \rho_{1100K})^2}}{1150-1100} [\frac{pcm}{K}] 
\end{align*}

Coolant reactivity coefficient (FLiBe): 
\begin{align*}
    \frac{\Delta \rho}{\Delta T_c} &= 
    \frac{\rho_{1150K}-\rho_{1100K}}{1150-1100} [\frac{pcm}{K}] \\
    \delta \frac{\Delta \rho}{\Delta T_c} &= 
    \frac{\sqrt{\delta (\rho_{1150K})^2+(\delta \rho_{1100K})^2}}{1150-1100} [\frac{pcm}{K}] 
\end{align*}

Graphite reactivity coefficient (graphite): 
\begin{align*}
    \frac{\Delta \rho}{\Delta T_g} &= 
    \frac{\rho_{1150K}-\rho_{1100K}}{1150-1100} [\frac{pcm}{K}] \\
    \delta \frac{\Delta \rho}{\Delta T_g} &= 
    \frac{\sqrt{\delta (\rho_{1150K})^2+(\delta \rho_{1100K})^2}}{1150-1100} [\frac{pcm}{K}] 
\end{align*}
We assumed all graphite. 

\subsection{Fission source distribution (c)}
Fission density (FD) is calculated by using openmc's `fission' score (f) divided 
by the average of all `fission' scores: 
\begin{align*}
    FD_i &=  \frac{f_i}{f_{ave}}
    \intertext{where:}
    f_{ave} &= \mbox{average of fission scores}
\end{align*}
The uncertainty calculations for $f_{ave}$ and $FD_i$: 
\begin{align*}
    \delta f_{ave} &= \frac{1}{N}\sqrt{\sum_i^Nf_i^2} \\
    \delta FD_i &= |FD_i| \sqrt{(\frac{\delta f_i}{f_i})^2+(\frac{\delta f_{ave}}{f_{ave}})^2}
    \intertext{where:}
    N &= \mbox{No. of fission score values} 
\end{align*}

\subsection{Neutron Flux (d, e, f)}
Openmc's `flux' score is given in units of [$\frac{n * cm}{src}$]. For the benchmark, 
we need to convert it to units of [$\frac{n}{cm^2s}$]. 
The conversion: 

\begin{align*}
    \Phi_c &= \frac{N* \Phi_o}{V} \\
    N &= \frac{P*\nu}{Q*k} \\
    \intertext{where:} 
    \Phi_c &= \mbox{Converted Flux [$\frac{neutrons}{cm^2s}$]} \\
    \Phi_o &= \mbox{Original Flux [$\frac{neutrons* cm}{src}$]} \\
    N &= \mbox{Normalization factor [$\frac{src}{s}$]} \\
    V &= \mbox{Volume of fuel assembly [$cm^3$]} \\
    P &= \mbox{Power [$\frac{J}{s}$]} \\
    \nu &= \mbox{$\frac{\nu_f}{f}$ [$\frac{neutrons}{fission}$]} \\
    Q &= \mbox{Energy produced per fission [$\frac{J}{fission}$]} \\
    &= \mbox{$3.2044*10^{-11}$ J per $U_{235}$ fission} \\
    k &= \mbox{$k_{eff}$ [$\frac{neutrons}{src}$]}
\end{align*}
Flux standard deviation: 
\begin{align*}
    \delta \Phi_c = \Phi_c * 
    \sqrt{(\frac{\delta \Phi_o}{\Phi_o})^2+ (\frac{\delta \nu_f}{\nu_f})^2 
    + (\frac{\delta k}{k})^2 + (\frac{\delta f}{f})^2}
\end{align*}
Reactor power is calculated based on the given reference specific power ($P_{sp}$) of 200 
$\frac{W}{gU}$. 
\begin{align*}
    P &= P_{sp} * V_F * \rho_F * \frac{wt\%_{U}}{100} 
    \intertext{where:}
    P &= \mbox{Reactor power [W]} \\
    V_F &= \mbox{Volume of fuel [$cm^3$]} \\
    &= \frac{4}{3} \pi r_1^3 * 101 * 210 * 4 * 2 * 6 * 3\\
    \rho_F &= \mbox{density of fuel [$g/cc$]} \\
    wt\%_{U} &= \frac{at\%_{U235} * AM_{U235} + at\%_{U238} * AM_{U238}}{
        \sum (at\%_i * AM_i)} * 100 \\
    AM &= \mbox{atomic mass}
\end{align*}

\section{Phase 1b}

\subsection{Depletion steps}
Openmc requires the user to input the time instances to conduct depletion till 
as opposed to the burnup value to conduct depletion till.
In the benchmark, the burnup value is specified, therefore, we need a method 
to determine how long the assembly must be depleted to reach the desired burnup. 

\begin{align*}
    T &= \frac{tU*BU}{P}
    \intertext{where:}
    T &= \mbox{depletion time [day]} \\
    tU &= \mbox{mass of enriched Uranium [metric tonnes]} \\
    BU &= \mbox{Burnup [MWd/t]} \\
    P &= \mbox{thermal power [MW]} 
\end{align*}

\subsection{Choosing depletion step size}
Openmc states: 
\begin{displayquote}
\textit{A general rule of thumb is to use depletion step sizes around 2 MWd/kgHM, 
where kgHM is really the initial heavy metal mass in kg. These are typically 
valid for the predictor scheme, as the point of recent schemes is to extend 
this step size. A good convergence study, where the step size is decreased until 
some convergence metric is satisfied, is a beneficial exercise.}
\end{displayquote}
The `maximum' depletion step:
\begin{align*}
    \Delta t_{max} &= \frac{2MWd/khHM*hm_{op}}{P}
    \intertext{where:}
    \Delta t_{max} &= \mbox{Maximum depletion step [days]} \\
    hm_{op} &= \mbox{mass of initial heavy metal [g]} \\
    P &= \mbox{Power [W]}
\end{align*} 

\subsection{Unit conversion}
After running depletion, openmc returns the total number of atoms for each nuclide. 
The benchmark requests results returned in relative density units [g/tHMi]. 
Unit Conversion: 
\begin{align*}
    \rho_{rel} &= \frac{A_{Ti} * amu}{NA * hm_{op}} 
    \intertext{where:}
    \rho_{rel} &= \mbox{Relative density [g/tHMi]} \\
    A_{Ti} &= \mbox{total no. of atoms per nuclide [atoms]} \\
    amu &= \mbox{atomic mass unit [g/mol]} \\
    NA &= \mbox{Avogadro constant [atoms/mol]} \\
    hm_{op} &= \mbox{mass of initial heavy metal [tHM]} 
\end{align*}


\end{document}
